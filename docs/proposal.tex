\documentclass[]{article}
\usepackage{lmodern}
\usepackage{amssymb,amsmath}
\usepackage{ifxetex,ifluatex}
\usepackage{fixltx2e} % provides \textsubscript
\ifnum 0\ifxetex 1\fi\ifluatex 1\fi=0 % if pdftex
  \usepackage[T1]{fontenc}
  \usepackage[utf8]{inputenc}
\else % if luatex or xelatex
  \ifxetex
    \usepackage{mathspec}
    \usepackage{xltxtra,xunicode}
  \else
    \usepackage{fontspec}
  \fi
  \defaultfontfeatures{Mapping=tex-text,Scale=MatchLowercase}
  \newcommand{\euro}{€}
\fi
% use upquote if available, for straight quotes in verbatim environments
\IfFileExists{upquote.sty}{\usepackage{upquote}}{}
% use microtype if available
\IfFileExists{microtype.sty}{%
\usepackage{microtype}
\UseMicrotypeSet[protrusion]{basicmath} % disable protrusion for tt fonts
}{}
\usepackage{longtable,booktabs}
\ifxetex
  \usepackage[setpagesize=false, % page size defined by xetex
              unicode=false, % unicode breaks when used with xetex
              xetex]{hyperref}
\else
  \usepackage[unicode=true]{hyperref}
\fi
\hypersetup{breaklinks=true,
            bookmarks=true,
            pdfauthor={},
            pdftitle={},
            colorlinks=true,
            citecolor=blue,
            urlcolor=blue,
            linkcolor=magenta,
            pdfborder={0 0 0}}
\urlstyle{same}  % don't use monospace font for urls
\setlength{\parindent}{0pt}
\setlength{\parskip}{6pt plus 2pt minus 1pt}
\setlength{\emergencystretch}{3em}  % prevent overfull lines
\setcounter{secnumdepth}{0}

\date{}

\begin{document}

\section{QUARK}\label{quark}

\subparagraph{QUantum Analysis and Realization
Kit}\label{quantum-analysis-and-realization-kit}

A High Level Programming Language for Quantum Computing

\subsubsection{Team}\label{team}

In lexicographical order:

\begin{longtable}[c]{@{}lll@{}}
\toprule
Name & UNI & Role\tabularnewline
\midrule
\endhead
Daria Jung & djj2115 & Verification and Validation\tabularnewline
Jamis Johnson & jmj2180 & System Architect\tabularnewline
Jim Fan & lf2422 & Language Guru\tabularnewline
Parthiban Loganathan & pl2487 & Manager\tabularnewline
\bottomrule
\end{longtable}

\subsubsection{Introduction}\label{introduction}

In the early 1980's, Richard Feynman observed that certain quantum
mechanical effects could not be efficiently simulated using classical
computation methods. This led to the proposal for the idea of a
``quantum computer'', a computer that uses the effects of quantum
mechanics, such as superposition and entanglement, to its advantage.
Though quantum computing is still in relative infancy, in 1994, Peter
Shor (Bell Labs) developed a quantum algorithm to factor integers in
polynomial time, providing motivation and renewed interest in building
quantum computers and discovering other quantum algorithms.

Classical computers require data to be encoded in binary digits, where
each bit is always in a definite state of either 0 or 1. Quantum
computation uses qubits in order to represent a superposition of states.
Operating on qubits effectively operates on different possible states of
being at the same time. By performing a single operation on a one bit
qubit, we perform operations on two different states of a qubit at once.
With certain algorithms, we can use this parallelism in order to solve
problems in significantly less time than a classical computer would
take.

We would like to propose QUARK, a domain-specific imperative programming
language to allow for expression of quantum algorithms. The purpose of
QUARK is to define quantum computing algorithms in a user-friendly way
and construct real quantum circuit instructions. In theory, our language
could produce quantum circuits that could be run on actual quantum
computers in the future. QUARK incorporates classical computing methods
and control structures, which allow operations on quantum and classical
data. It supports user defined operators and functions to make the
esoteric notion of writing algorithms for quantum computers a reality.
Built in data types like complex numbers, fractions, matrices and
quantum registers combined with a robust standard library that supports
Hadamard and other quantum gates make QUARK a great way to get started
with quantum computing.

A basic quantum circuit simulator is included as part of the QUARK
architecture. Quantum operators and data types in QUARK compile to C++,
which can then be passed onto our quantum simulator.

\subsubsection{Syntax}\label{syntax}

\subparagraph{Comments}\label{comments}

\begin{verbatim}
% single line comment
%{
  multi-line comments
}%
\end{verbatim}

\subparagraph{Variable Declarations}\label{variable-declarations}

Variables are declared in an imperative style with dynamic typing. There
is no need to declare the type of the variable or demarcate a new
variable with a keyword. The variable name is on the left and it is
assigned a value using \texttt{=} operator to the result on the right
side of the assignment. Also, every line ending is indicated by a
\texttt{;} like in Java or C. We also suggest naming variables using
underscores.

\begin{verbatim}
some_variable = "variable";
some_other_variable = 10;
\end{verbatim}

\subparagraph{Types}\label{types}

QUARK supports the following types: - Numbers - Fractions - Complex
Numbers - Booleans - Strings - Lists - Matrices - Quantum Registers

Numbers

All numbers are floats. There is no distinction between integers and
floats.

\begin{verbatim}
num = 2;
pi = 3.14;
\end{verbatim}

Fractions

Fractions can represent arbitrary precision and are represented using a
\texttt{\$} sign to separate the numerator and denominator.

\begin{verbatim}
% pi represents 22/7
pi = 22$7;
1/pi; % returns 7/22
pi + 1$7; % returns 23/7
\end{verbatim}

Complex Numbers

Complex numbers can be represented using the notaiton \texttt{a+bi}
where \texttt{a} and \texttt{b} are Numbers.

\begin{verbatim}
complex = 3+1i % This represents 3+i. We need b=1. It can't be omitted
complex * -.5i; % Arithmetic operations on complex numbers. This returns .5-1.5i
(-.5 + .3i) ** 5; % Use **n to raise to the power n
complex = 2.6 - 1.3i;
% The following assertions reurns true
norm(complex) == 2.6**2 + (-1.3)**2; % use norm() to get the norm
complex[0] == 2.6; % get real part
complex[1] == -1.3; % get imaginary part
abs(complex) == sqrt(norm(complex)); % use abs() to get the absolute value
\end{verbatim}

Booleans

It's simply \texttt{true} and \texttt{false}.

\begin{verbatim}
is_this_an_awesome_language = true;
\end{verbatim}

Strings

Strings can be represented within double quotes. There are no
characters. Characters are just strings of length 1. Escape a double
quote with \texttt{\textbackslash{}} as in \texttt{\textbackslash{}"}.
Get the string length with the \texttt{len()} function. Concatenate
strings with \texttt{+=}. Access parts of string using \texttt{{[}{]}}.

\begin{verbatim}
some_string = "Hello World";
len(some_string); % returns 11
some_string += "!"; % It's now "Hello World!"
some_string[4]; % returns "o"
\end{verbatim}

Lists

We make it easy to use lists, kind of like Python.

\begin{verbatim}
some_list = [1:5]; % returns list {1, 2, 3, 4, 5};
another_list = {"a", "b", "c", "d", "e", "f"}; % can be used to explicitly define elements in list
another_list[-1]; % returns "f"
another_list[2:4]; % returns {"c", "d", "e"}
len(lis); % len() returns the size of the first dimension of the list
\end{verbatim}

Matrices

Matrices are also easy to use. Similar to Matlab.

\begin{verbatim}
mat = [[2, 3],[5, 6],[-1, 2]];
mat'; % transpose a 2D matrix
mat2 = [[-2, 3],[0, 6]];
len(mat[0]) == 2; % returns true. This is the column dimension
len(mat) == 3; % returns true. This is the row dimension
mat[0][1]; % get element at position (0,1)
\end{verbatim}

Quantum Registers

Quantum registers are the essential containers for qubit states and
entanglement. Our language (and simulator) supports two modes of a
quantum register: dense and sparse mode. Note that they are used for
simulation only. Please use sparse mode if you know in advance that your
quantum state vector will have relatively few non-zero entries.

Measuring a register, partially or totally, will force certain states to
collapse probabilistically. A measurement is a non-reversible intrusive
operation on a physical quantum computer, but our simulator supports an
unrealistic mode of repeated measurement to facilitate simulation.

The measurement operator is \texttt{?}, while the unrealistic
non-destructive measurement is \texttt{?\textquotesingle{}}

\begin{verbatim}
size = 5;
% quantum register  construction
qr1 = <size, 1>; % dense quantum register with initial state 1
qr2 = <size, 0>'; % sparse quantum register with initial state 0

% measurement
q ? 1; % measure a single qubit. Returns either 0 or 1
q ? 2:5; % measure qubit 2 to 4. Returns an integer from 0 to 31 that represents the resultant state.
q ? :5; % from qubit 0 to 4.
q ? 3:; % from qubit 3 to the last qubit. 
q ?; % measure the entire register. The result will range from 0 to 2^size - 1

%{
same as above, except that you can repeatedly measure 
without disrupting the quantum states. 
Use this mode with caution because it is unrealistic.
}%
q ?' 1;
q ?' 5:;
q ?';
\end{verbatim}

\subparagraph{Control Flow}\label{control-flow}

if

\texttt{if} behaves similar to C. If the body of \texttt{if} has only
one line, the brackets are optional. The condition is terminated by a
colon.

\begin{verbatim}
l = [1, 2];
if len l == 0 :
    return l;

if 1 == 1 :
{
    x = 2;
    x += 1;
}
else
{
    x = 6;
    x ++;
}
\end{verbatim}

for

Our \texttt{for} is similar to python's for

\begin{verbatim}
for number in numbers:
    number += 1
\end{verbatim}

Example of iterating over half of a list

\begin{verbatim}
for val in list[0:len(list)/2]:
    val = val*10;
\end{verbatim}

You can also iterate over qubits in a quantum register

\begin{verbatim}
q = <10, 0>
for qbit in [: len(q)]:
    had q qbit;
\end{verbatim}

while

\texttt{while} is python style and continually runs the block until the
boolean expression becomes false

\begin{verbatim}
while val != 0 :
    val -= 1;
\end{verbatim}

Function definition

\texttt{def} keyword defines a new function. The parentheses at the line
of \texttt{def} are optional.

Optional arguments are denoted by \texttt{arg\_name = default\_value}

\begin{verbatim}
def f1 a, b = 3
{
    c = a ** b + 1i;
    return c * (2 - 3i) + a;
}
\end{verbatim}

Lambda

There are two styles of lambda.

The short version has a single statement as its body:

\begin{verbatim}
square = lambda x : x * x;
\end{verbatim}

The long version has its body enclosed in brackets and can have more
complicated control flow. A \texttt{return} statement is required if the
lambda wants to return a non-void value.

\begin{verbatim}
square = lambda x : { return x * x; };
(lambda x : {if x < 2: return -1; else return 100;})(30) % return 100
\end{verbatim}

\subparagraph{Built-in Functions}\label{built-in-functions}

bit

Viewing the bit value of an integer is also critical to QC. \texttt{bit}
takes an integer and bit position. The position is counted from the
least significant bit.

\begin{verbatim}
bit(15, 0); % returns 1
\end{verbatim}

len

\texttt{len} is short for length. \texttt{len} returns the length of a
list, the number of characters in a string, the number of qubits in a
quantum register, or the number of rows in a matrix.

\begin{verbatim}
list = [1,2,3,4,5];
len list; % returns 5
str = "hi earth";
len str == 8; % returns true
q_reg = <10,0>;
len(q_req); % returns 10
mat = [[1,2,3],[4,5,6]];
len mat; % returns 2
\end{verbatim}

\subsubsection{Sample Code}\label{sample-code}

\subparagraph{Classical algorithms}\label{classical-algorithms}

gcd algorithm

\begin{verbatim}
def gcd(a, b)
{
    while a != 0:
    {
        c = a;
        a = b mod a;
        b = c;
    }
    return b;
}
\end{verbatim}

Bitwise dot product

\begin{verbatim}
def bit_dot a, b
{
    limit = 1;

    c = a & b;

    counter = 0;
    i = 0;

    while limit <= c:
    {
        counter += bit c i++;
        limit <<= 1; % same as limit *= 2
    }

    return counter mod 2;
}
\end{verbatim}

\subparagraph{User-defined quantum
gates}\label{user-defined-quantum-gates}

Multi-qubit hadamard gate operation

\begin{verbatim}
def had_multi(q, lis)
{
    %{ if the list is empty, 
      we apply had() to all bits
    }%

    if len lis == 0:
        lis = [: len q];

    for i = lis:
        had q, i;
}


def had_range(q, start, size = 1)
{
    had_multi q, [start: start+size] ;
}
\end{verbatim}

Quantum Fourier Transform

\texttt{qft\_sub} is the recursive subroutine used by \texttt{qft}

\begin{verbatim}
def qft_sub q, start, size
{
    if size == 1:
    {
        had(q, start);
        return;
    }

    % recurse
    qft_sub(q, start, size-1);

    last = start + size - 1;
    for t = [start : last]:
    % control gates are prefixed with 'c_'
        c_phase_shift q, PI / 2**(last - t), last, t;

    had q, last ;
}

def qft(q, start = 0, size = len q)
{
    qft_sub q start size;

    % reverse the qubits
    for tar = [start : start + size/2]:
        swap q, tar, tar+tarSize-1-tar;
}
\end{verbatim}

\subparagraph{Sample quantum
algorithms}\label{sample-quantum-algorithms}

Almost all quantum algorithms consist of a classical part and a quantum
part. Classical part typically involves pre- or post-processing on a
normal computer. Quantum part involves qubits and quantum circuits.
\#\#\#\#\#\# Deutsch-Josza Parity algorithm An efficient O(1) quantum
algorithm to solve the Deutsch-Josza parity problem. The theoretical
lower bound of a classical algorithm for this problem is O(n).

\begin{verbatim}
import myutil; % user-defined libraries
import mygate;

% keyboard input
secret = int(
    input("Secret 'u' for Deustch_Josza parity algorithm"));

% quantum oracle
ocfun = lambda x : bit_dot(x, secret);

nbit = 5;

% dense mode, 6 qubits, initialize to state 1
q = <nbit+1, 1>;

% or sparse mode. 
q = <nbit+1, 1>';

% apply hadamard gates
for i = [: len q] :
    had q, i;

% apply oracle
oracle q, ocfun, nbit;

% using a library function from mygate.qk
had_multi q, 0, nbit;

% measurement
result = q ? 0:nbit; 
\end{verbatim}

Grover's Search

This is one of the most celebrated quantum algorithms ever invented.
Grover's search can efficiently find the needle in an unsorted haystack
in O(sqrt(N)) time, while the trivial lower bound for classical search
algorithms is O(N).

\begin{verbatim}
key = int(input("The key to search"));

% the search oracle
ocfun = lambda x : { return x == key; };

nbit = 5;
N = 2 ** nbit;
q = <nbit, 0>;

for i = [:nbit] :
    had q, i;

for iter = [: floor(sqrt N)]:
{
    % apply search oracle
    oracle q, ocfun, nbit;

    had_multi q, [:nbit];

    % define a diffuse matrix
    % initialize to all zeros
    diffuse = zeros N, N;
    diffuse[0][0] = 1;
    for i = [1: len diffuse]:
        diffuse[i][i] = -1;

    % apply this unitary gate
    generic_gate q, diffuse;

    had_multi q, [:nbit];
}

% measure the whole register, 
% then shift one bit to the right.
result = (q ?) >> 1;
\end{verbatim}

\end{document}
